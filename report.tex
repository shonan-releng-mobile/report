\documentclass[a4paper]{article}
\usepackage{graphics}
\usepackage[dvipdfmx]{graphicx}
\usepackage{bmpsize}
\usepackage{shonan}
\usepackage{hyperref}
\usepackage{enumerate}
\usepackage{tabularx}
\usepackage{url}

\newcommand{\smallsection}[1]{\noindent \textbf{#1}. }

\begin{document}

% Generate a standard cover of an NII Shonan Meeting Report.
\SHONANno{152}
\SHONANtitle{Release Engineering for Mobile Applications}
\SHONANauthor{%
Shane McIntosh\\
Yasutaka Kamei\\
Meiyappan Nagappan}
\SHONANdate{December 09--12, 2019}
\SHONANmakecover

\def\tightlist{\itemsep1pt\parskip0pt\parsep0pt}

\title{Release Engineering for Mobile Applications}
\author{Organizers:\\
Shane McIntosh (McGill University, Canada)\\
Yasutaka Kamei (Kyushu University, Japan)\\
Meiyappan Nagappan (University of Waterloo, Canada)}
\date{December 9--12, 2019}
\maketitle

\section{Overview of the Meeting}

\subsection{Background}

Release engineering is the process that ships code changes from a developer's workspace to the end user.
It encompasses tools and technologies to accomplish continuous integration, deployment, delivery, and experimentation, such as build specifications, infrastructure-as-code, containerization, and many more.

In recent years, there has been a renewed interest in this area, driven by the need to deliver new content to users in a continuous fashion.
For example, popular web browsers like Mozilla Firefox and Google Chrome have shifted from slow, semi-annual releases to rapid, six-week release cycles.
In the domain of web applications, where control over the main delivery process is retained, industry has produced innovative techniques to achieve continuous delivery of new content.
These techniques allow for new content to be tentatively released (e.g., canary deployment), seamlessly released (or rolled back) without customer-facing downtime (e.g., blue-green deployment), and analytically evaluated with data from the field (e.g., A/B testing).

At the same time, mobile applications have become a typical means of delivering content to users. For example, recent market studies show that about four billion mobile devices are connected in 2017\footnote{\url{https://is.gd/OVPgze}} and predict that the global mobile app economy is expected to be worth \$157 billion by 2022.\footnote{\url{https://is.gd/61uIoy}}

\subsection{Challenges and Opportunities}

Rather than delivering content directly through the web browser, mobile app users install applications on their mobile devices (e.g., smartphones, tablets).
Release engineers need to deliver new releases from a software organization to app stores that curate mobile apps and make them available to users.
The existence of third-party app stores in the release path from software organization to user poses several challenges:

\begin{itemize}
\item Users must opt-in to an update of the client-side app in order for new content to be delivered. Releasing too quickly creates an overhead for users, since downloading and installing updates may slow down access to apps. Moreover, with the explosive growth of mobile app sizes, users may need to pay expensive data fees for large downloads on cellular networks.
\item App stores act as a bottleneck between software organizations and users, which may impose delays in the release process. For example, to prevent malware and exploit- vulnerable apps from impacting users, all of the apps that appear in the Apple app store need to be certified by an Apple technician. This certification process is slow, cumbersome, and expensive.
\end{itemize}

\subsection{Goals of the Meeting}

The purpose of this Shonan meeting is to bring together leading researchers and practitioners from the release engineering and mobile apps fields. The aim is not only to foster an exchange of ideas, but also to outline a clear set of concrete grand challenges and propose a roadmap for how those challenges can be met by future work.

A tangible outcome that we are aiming to produce is a manuscript that describes the grand challenges and our roadmap for future research.
For example, a foreseeable grand challenge is how software organizations can achieve crowd-based deployment strategies that have become so popular for web applications (e.g., canary and blue-green deployment) in a mobile setting.
The produced manuscript will be submitted as a vision paper to a conference or journal.

\clearpage

\section{Meeting Schedule}

\begin{bfseries}
Check-in Day: December 08 (Sun)
\end{bfseries}
\begin{itemize}
\item Welcome Banquet
\end{itemize}
\begin{bfseries}
{Day1: December 09 (Mon)}
\end{bfseries}
\begin{itemize}
\item Plenary: Introduction to the meeting
\item Plenary: Introductory lightning talks
\item Breakout: Breakout session 1 (Artefacts, Fragmentation, Human Aspects, Log Analysis)
\item Plenary: Reports and discussion on breakout session 1
\end{itemize}
\begin{bfseries}
Day2: December 10 (Tue)
\end{bfseries}
\begin{itemize}
\item Plenary: Planning for breakout session 2 
\item Breakout: Breakout session 2 (Definitions)
\item Plenary: Reports and discussion on breakout session 2
\item Plenary: Planning for breakout session 3
\item Breakout: Session 2 (Continuous Experimentation, Quality Assurance, Tools)
\item Plenary: Reports and discussion on breakout group 3
\item Demo: Julian Hardy
\end{itemize}
\begin{bfseries}
Day3: December 11 (Wed)
\end{bfseries}
\begin{itemize}
\item Plenary: Plan for breakout session 4
\item Breakout: Session 4 (Seeding Collaborative Projects)
\item Group photo shooting
\item Excursion and Main Banquet
\end{itemize}
\begin{bfseries}
Day4: December 12 (Thu)
\end{bfseries}
\begin{itemize}
\item Plenary: Reports and discussion from breakout session 4
\item Breakout: Session 5 (Solidifying Collaboration Plans)
\item Plenary: Meeting retrospective \& closing
\end{itemize}

\clearpage

\section{Plenary Talks}

\subsection{Walkthrough of Modern Release Analyics Workbench for Android Apps}
\smallsection{Authors}
Julian Harty, Commercetest Limited/Open University, UK \linebreak

The walkthrough was a live demonstration by Julian Harty focusing primarily on Google Play Console including two integrated facilities: 1) Release Engineering and 2) Android Vitals analyics and reporting. We also performed very brief walkthroughs of the continuous build processes for several Android apps.

\subsubsection{Google Play Console}
Google Play Console is the interface Google provides primarily for developers of Android apps in the Google Play Store. It is intended to provide developers with tools, reports, and insights about how their apps are performing in terms of pupularity, usage, revenues, stability, ratings and reviews. Developers (with appropriate permission) can also manage releases from both a testing and in terms of managing production releases.

Each app belongs to an owner, who will have registered as a developer and paid a one-time fee. Owners can share aspects of their account with other people who have a Google account and who have accepted Google's terms and conditions, etc. The sharing can be granular and limited to a single app owned by a particular owner, the owner can also set the permissions that these other people will have for one or more apps they own. Individuals can have access to multiple accounts and distinct permissions for each app and/or account.


\subsubsection{The apps}
Julian Harty has been involved in each of the following projects and therefore able to introduce the tools used by each project and aspects of the development, testing, and release processes. In each case there are lead developers for the particular who may be willing to provide further details and insights related to Release Engineering.

\begin{itemize}
\item Android Apps from the catrob.at team at TU Graz (source code   [\href{https://github.com/catrobat/}{https://github.com/catrobat/}], project homepage [\href{https://www.catrobat.org/}{https://www.catrobat.org/}], JIRA [\href{https://jira.catrob.at/}{https://jira.catrob.at/}], CI [\href{https://jenkins.catrob.at/}{https://jenkins.catrob.at}])

\item Android Apps from the Kiwix team (source code [\href{https://github.com/kiwix/kiwix-android/}{https://github.com/kiwix/kiwix-android/}], project homepage [\href{https://www.kiwix.org/}{https://www.kiwix.org/}]). (note: currently migrating from travis-ci - Kiwix on Travis-CI [\href{https://travis-ci.com/github/kiwix/kiwix-android}{https://travis-ci.com/github/kiwix/kiwix-android}] to GitHub Actions -  GitHub Actions for Kiwix [\href{https://github.com/kiwix/kiwix-android/actions}{https://github.com/kiwix/kiwix-android/actions}]).

\item Android Apps from the eduVPN project (source code [\href{https://github.com/eduvpn/android}{https://github.com/eduvpn/android}], project homepage [\href{https://eduvpn.nl/}{https://eduvpn.nl/}]. A proof-of-concept CI is on Travis-CI [\href{https://travis-ci.com/github/commercetest/android}{https://travis-ci.com/github/commercetest/android}]
\end{itemize}

\subsubsection{CI and CB}
\textbf{Characteristics of the Build Process for these apps}: Each project team has a distinct build process. Of these, the build process and tooling for the Catrobat apps is the most sophisticated and mature. 

\textbf{Kiwix}: The Kiwix Android codebase incorporates a library written in C++ that is shared across many Kiwix projects (including web servers and the iOS app). There are distinct codebases, on GitHub, for the library and for shared build processes and tools. Development builds use pre-built binaries for the native code. 

\textbf{Catrobat apps}: The Catrobat apps are built using a farm of Jenkins servers. 

\textbf{eduVPN}: The app is independently built by a developer who does not work directly on the development of the app's codebase. 

\subsubsection{Characteristics of testing for these apps}
\textbf{Kiwix}: Until the end of 2019 the project used a combination of Travis-CI and TestDroid to run the automated tests on several hosted physical devices.

\textbf{Catrobat apps}: Automated tests are run by Jenkins. 

\textbf{eduVPN}: Testing is ad-hoc and best described via one of the project's 'issues': eduVPN issue 221 [\href{https://github.com/eduvpn/android/issues/221}{https://github.com/eduvpn/android/issues/221}]


\subsubsection{Characteristics of the Release Engineering for these apps}

\textbf{Elapsed times}

\begin{itemize}
\item Kiwix app
\item Custom Kiwix apps: WikiMed, ...
\item eduVPN: eduVPN, Home edition (not currently in a CB)
\item Catrobat: PocketCode, PocketPaint
\end{itemize}

\subsubsection{Glossary}

\begin{itemize}
\item CB: Continuous Build
\item CI: Continuous Integration
\end{itemize}
\clearpage

\section{Breakout Group Discussions}

\subsection{Session 1: Exploring Topics of Interest}

\subsubsection{Artefacts for Release Engineering of Mobile Apps}

\smallsection{Participants/Authors}
Cuiyun Gao, Yasutaka Kamei, Li Li, Shane McIntosh, Sebastian Proksch

\smallsection{Discussion Points}
The problems discussed are not specific to mobile app stores, rather than to generic centralized app stores (like Steam Game Platform, MS Windows Appstore, etc.). We think that, to draw the line between an App Store and a regular package manager is the ability of users to review and rate the individual store elements.

\smallsection{Data Sources}
\begin{itemize}
    \tightlist
    \item User facing
        \begin{itemize}
            \tightlist
            \item social media (reddit, blogs, twitter, FB, forums)
        \end{itemize}
    \item Developer facing
        \begin{itemize}
            \tightlist
            \item App Stores
                \begin{itemize}
                    \tightlist
                    \item Apple, Google, Steam (commercial)
                    \item F-Droid (probably not representative), Linux Package Managers ... (open source)
                \end{itemize}
            \item app releases (app lineage)
            \item commit history
            \item StackOverflow Discussion
            \item Android Framework Evolution
        \end{itemize}
    \item In between
        \begin{itemize}
            \tightlist
            \item Logs
            \item Crash Reports
        \end{itemize}
\end{itemize}

\smallsection{Pipeline} Table \ref{table:release} shows the artifacts in each phase of release pipeline. The bold phases are the ones that can benefit most from the use of social media. We can use posts as indicators of the release process or to mine new requirements. The other phases are a black box for commercial apps. It is hard to study these, because we lack access to reliable sources.

In general, several research challenges have to be solved, like how to automatically map social media posts and apps.

\begin{table}[h] \label{table:release}
    \centering
    \caption{Release Pipeline and artifacts}
    \begin{tabular}{l|l}
        \hline
        RelEng Phase & Artifacts \\ \hline 
        \textbf{Requirements Engineering} & social media, user         reviews \\ 
        Integration & Open Source App Stores (e.g., F-Droid) \\
        Build + Test & Git, CI Providers \\
        Deployment & app lineage \\
        \textbf{Monitoring + Reaction} & logs, social media, user
        reviews  \\ \hline
    \end{tabular}
\end{table}

\subsubsection{Overcoming Fragmentation of Android Versions in Deployment of Mobile Apps}

\smallsection{Participants/Authors} Carmine Vassallo, Keheliya Gallaba, Raula Gaikovina Kula, Li Li, Daniel Dominguez

\smallsection{Definitions} Push-based deployment model (in case of a web app): Build $\rightarrow$ Deploy $\rightarrow$ Release

Pull-based deployment model (in case of mobile apps): A pull request is generated when the app is submitted to the App Store. The App store can refuse/accept it as well as the user.

\smallsection{Challenges} 

\begin{itemize}
    \tightlist
    \item For developers
        \begin{itemize}
            \tightlist
            \item Misalignment on Frontend \& Backend
            \item Traceability of app reviews to an app version (and therefore to a changelog)
            \item Traceability between user and developer artifacts
            \item Can not force users to update
            \item Support legacy software
        \end{itemize}
    \item For researchers
        \begin{itemize}
            \tightlist
            \item Lack of data for studies. In particular backend information
        \end{itemize}
\end{itemize}

\smallsection{Research directions}
\begin{itemize}
    \tightlist
    \item Feature toggle updates to emulate canary updates (How to get them right and proper)
    \item Factors that drive changes and updates to the users (What is a good release note that make people want to update faster?)
    \item Usage of network logs for approximating the backend behavior Compare the behavior of the webapp to the frontend against the same backend
\end{itemize}


\subsubsection{Human Aspects of Release Engineering for Mobile Apps}

\smallsection{Participants/Authors}
Tom Zimmermann, Daniel German, Mike Godfrey, Fabio Palomba, Pick Thongtanunam, Kla Tantithamthavorn, Toshiki Hirao, Al-Subaihin, Masanari Kondo

\smallsection{Contents (Figure \ref{figHumanAspects})}
Release engineering is not about bug fixes, but it's about social contract / interactions between users and developers. We break users into 2 types, i.e., new users and existing users. Are we in the age of release economics? We try to release apps to get more money (\$\$\$), get more users, and change users' behaviors. For example, we are about to release a new version with bug fixes and updates but don't forget to pay \$5. So, we hypothesize that users can change release engineering, and release engineering can change users' behaviors.

\begin{figure}[h] \label{figHumanAspects}
  \centering
  \includegraphics[width=0.7\columnwidth]{fig/day1.jpg}
  \caption{Brainstorming session for human aspects of release engineering for mobile apps}
\end{figure}

\begin{itemize}
    \tightlist
    \item Q1: How do users accept releases (1st release for new users and later for existing users)? is it based on automatically updates?
    \item Q2: How do developers positively and negatively change the users' behaviors.
    \item Q3: How do app stores / ecosystems allow dev/users to accept releases?
    \item Implications/Benefits: win-win strategies for both users and developers + anti-patterns in release engineering.
    \item Other questions:
        \begin{itemize}
            \tightlist
            \item Why mobile apps do not use issue tracking systems? Why do App stores do not provide ITSs for users?
            \item Do developers care about bugs?
            \item How do users accept releases?
            \item Do users care about security? What would users do or take to care about security? What are security-critical apps? (not all apps/users need to care about security)
            \item Do different app stores have different users' expectation / users' behaviors? (games have high reviews than educational app reviews)
            \item Why do the same apps but in different stores have different price?
            \item Do users from different app stores have different expectation/behaviors?
        \end{itemize}
\end{itemize}

\subsubsection{Log Analysis for Mobile Apps}

\smallsection{Participants/Authors}
Meiyappan Nagappan, Julian Harty, Maur\'{i}cio Aniche, Luca Pascarella, Weiyi Shang

\smallsection{Overview}
This document discusses:

\begin{itemize}
\tightlist
\item
  Empirical studies in log engineering that we should do
\item
  (Existing) techniques we need to study and develop
\item
  Challenges faced by researchers in mobile log research
\item
  Existing tools/codebases to support researchers
\item
  Overall concerns in mobile logging
\end{itemize}

Our focus is on logging in the apps (the front-end) unless otherwise
indicated.

\smallsection{Empirical studies in log engineering}

\begin{itemize}
\tightlist
\item
  Why is logging used? Why do mobile devs log? Which logging libraries
  are used in mobile apps?
  \begin{itemize}
    \tightlist
    \item \url{https://users.encs.concordia.ca/~shang/pubs/Zeng2019_Article_StudyingTheCharacteristicsOfLo.pdf}
  \end{itemize}  
\item
  Performance overhead: how much logging is too much?
  \begin{itemize}
    \tightlist
    \item Signal-to-noise levels in the log messages?
    \item How to determine the frequency of the logs being delivered back to the developers.
  \end{itemize}
\item
  What do mobile developers use logs for?
  \begin{itemize}
    \tightlist
    \item e.g., Are logs being used for performance monitoring?
    \item Why do developers (even) use logs in apps where they cannot read the logs
  \end{itemize}
\end{itemize}

\smallsection{Techniques to be studied and developed}

\begin{itemize}
\tightlist
\item
  How much existing techniques should be changed/evolved for mobile
  apps? Do we need to devise specific tools for mobile?
  \begin{itemize}
    \tightlist
    \item Log analysis research currently focus on Anomaly detection, Security and Privacy, Root cause analysis, Failure prediction, Software testing, Model inference and invariant mining, Reliability and dependability
  \end{itemize} 
\item
  How to build different anomaly detection (ML) models for different
  types of user behavior?
    \begin{itemize}
    \tightlist
    \item In enterprise systems, we have been observing that a single model is not able to capture all the different behaviors of the different
  companies. We expect the same for mobile apps.
  \end{itemize}  
\item
  Can we automatically cluster the different user behaviors?
    \begin{itemize}
    \tightlist
    \item How do we cope with different versions? Maybe Robust Log-Based Anomaly Detection on Unstable Log Data by
    Zhang et al. might help.
  \end{itemize}  
\item
  When do we want to send the logs back to the cloud? And how?
    \begin{itemize}
    \tightlist
    \item Problems: cost, availability/connectivity, energy consumption.
    \item Companies like Crashlytics, and Splunk are being by developers.
    \end{itemize}
\item
  How do deal with the inconsistencies and the ever evolving log code
  base?
    \begin{itemize}
    \tightlist
    \item Robust Log-Based Anomaly Detection on Unstable Log Data. Xu Zhang,
  Yong Xu, Qingwei Lin, Bo Qiao, Hongyu Zhang, Yingnong Dang, Chunyu
  Xie, Xinsheng Yang, Qian Cheng, Ze Li, Junjie Chen, Xiaoting He,
  Randolph Yao, Jian-Guang Lou, Murali Chintalapati, Furao Shen, and
  Dongmei Zhang
    \end{itemize}
\item
  Can we introduce dynamic logging? I.e., logging only when it's really
  needed. Saves energy, bandwidth.
    \begin{itemize}
    \tightlist
    \item Can we send a remote command to start logging that area? i.e., only log when it's needed? Remember that you can't deploy your app every second, so dynamically activating logs might be a specific problem of mobile apps.
    \end{itemize}
\item
  Security of the logs: before shipping the logs to the cloud, maybe any
  app can access those logs. How to store them in a secure way?
\end{itemize}

\smallsection{Overall concerns}
\begin{itemize}
\tightlist
\item
  Avoid privacy leaks
  \begin{itemize}
\item
  Facebook ad library bug where the facebook token was written to the
  log, other apps can read
\item
  GDPR in EU.

  \begin{itemize}
  \tightlist
  \item
    Who is liable to check if the logging/log files in the apps are
    compliant?
  \item
    If you use your `own logging library to collect data', do you have
    to ask permission for the user? (Note that for logs you get in the
    app store, directly from Google, you don't need, as Google already
    asked for it)
  \end{itemize}
\item
  As a user, do you know what's being logged about you?
  \end{itemize}
\item
  Size of the applications. Mobile apps are often smaller than
  enterprise traditional projects. Mobile apps focus on single tasks
\item
  Is scalability as big of an issue as in other domains, e.g.,
  enterprise applications?
\item
  Who is responsible to make sure that the log frameworks are good-faith
  actors who do not have backdoors in the logging.
\item
  How much of the localization of log lines impact the tools and
  techniques?
\end{itemize}

\smallsection{Research challenges}

\begin{itemize}
\tightlist
\item
  We need log datasets from mobile applications
    \begin{itemize}
  \tightlist
\item
  LogPai benchmark: \url{https://github.com/logpai}, but it does not
  consider mobile apps.
\end{itemize}

\item
  Representative datasets of mobile apps source code

    \begin{itemize}
  \tightlist
\item
  FDroid might be not that representative\ldots{}
\end{itemize}
\end{itemize}

\smallsection{Existing tools}
Note: \href{https://github.com/ISNIT0}{ISNIT0} is Joseph Reeve's Github
account. Joe works with \href{https://github.com/julianharty}{Julian
Harty} on creating code to help research logging, mobile analytics.

\begin{itemize}
\tightlist
\item
  \url{https://github.com/ISNIT0/AndroidCrashDummy} a small
  exploratory/demo app to test/exercise other aspects of logging and
  analytics
\item
  \url{https://github.com/ISNIT0/AndroidLogAssert} a Log Assertion
  Library for use with Android Espresso tests
\item
  \url{https://github.com/ISNIT0/logcat-filter} Logcat-filter and
  analysis tool
\item
  \url{https://github.com/ISNIT0/log-searcher} A tool for searching
  Android codebases and analysing usage of ``Log.*''
\item
  \url{https://github.com/ISNIT0/log-complexity-comparison} This tool
  helps find complex code that does not have much logging to back it up.
  It was originally built to help with research done by Julian Harty and
  Joseph Reeve on logging placement. The output is a JSON file and an
  HTML report, describing which files need more logging attention.
\end{itemize}

\smallsection{Logging by the Operating System}
Google Android includes logging at the device level. Users have the
option to opt-in/out to automatically provide usage and diagnostics data
to Google {[}Play{]}. This logging includes usage, crashes, ANRs and
performance issues.

\subsection{Session 2: Establishing Working Definitions}
In this session, we discuss the same topic about ``What Makes Mobile Special for Releng?'', but make three small groups to ensure that different voices and opinions are heard. 

\subsubsection{Group 1. What Makes Mobile Special for Releng?}
\smallsection{Participants/Authors}
Daniel M. German, Shane McIntosh, Thomas Zimmermann, Li Li, Keheliya Gallaba

\begin{figure}[h] \label{figMobile}
\centering
\includegraphics[width=0.7\columnwidth]{fig/mobile.png}
\caption{Overview}
\end{figure}

\smallsection{Discussion (Figure \ref{figMobile})}
Lines are blurry about what is a mobile device.

\begin{itemize}
\tightlist
\item
    Is a chromebook with android apps a mobile device?
\item
    Is it a device designed for mobility?
\item
    Is it a device with limited hardware? Even defining what's mobile is a
challenging topic.
\end{itemize}

Then we narrow down on two dimensions on release engineering challenges
for mobile.

\begin{itemize}
\tightlist
\item
\textbf{Approval/Post-approval}: Centralized app stores
(gate keepers) Not necessarily mobile
\item
\textbf{Pre-approval}: Different
mobile device fragmentation based challenges (for development and
testing)
\end{itemize}

\smallsection{Approval/ post-approval challenges}

\begin{itemize}
\tightlist
\item
  Shares challenges with other gate-keeper based eco systems

  \begin{itemize}
  \tightlist
  \item
    steam game store, atlassian appstore etc
  \end{itemize}
\item
  Not only technical issues. Social/economical/policy challenges
\item
  Multiple gate keepers (company + government)
\item
  Different customer bases
\item
  How to make more money? recurring subscription or one time (different
  business models)
\item
  How to make users install/updates
\end{itemize}

\smallsection{Pre-approval challenges}

\begin{itemize}
\tightlist
\item
  Development is done in a different machine from the one it's deployed
\item
  Fragmented by different OSes, hardware, testing challenges (runtime)
\item
  Variants in devices
\item
  Testing for low bandwidth / limited hardware
\item
  \textbf{But main challenge: As researchers, we don't really know
  what's happening here. (we don't have access to this information)}
\end{itemize}

\subsubsection{Group 2. What Makes Mobile Special for Releng?}
\smallsection{Participants/Authors}
Afnan, Mauricio, Mike, Julian, Toshiki, Raula, Sebastian, Pick, Yasu

\smallsection{History}
The mobile developer's guide to the galaxy charts the history of
developing mobile apps. The 18th edition is currently online at \url{https://www.open-xchange.com/resources/mobile-developers-guide-to-the-galaxy/}

\smallsection{Discussion}\label{discussion}

\begin{figure}[h]
\centering
\includegraphics[width=0.7\columnwidth]{fig/convergence_with_mobile.jpg}
\caption{Convergence with Mobile}
\end{figure}

\begin{itemize}
\tightlist
\item
  Many open questions:

  \begin{itemize}
  \tightlist
  \item
    Do developers have different technical concerns?
  \item
    Have the stakeholders changed?
  \item
    What is mobile?

    \begin{itemize}
    \tightlist
    \item
      Is it something with limited hardware, or bandwidth?
    \item
      Or is a laptop a mobile device?
    \end{itemize}
  \end{itemize}
\item
  Clear challenges:

  \begin{itemize}
  \tightlist
  \item
    Gate keeper
  \item
    Deployment
  \item
    Feedback
  \end{itemize}
\item
  The future is the convergence!

  \begin{itemize}
  \tightlist
  \item
    Desktops will adopt a lot of the ideas from the mobile world
  \item
    The differences will get smaller and smalle
  \end{itemize}
\end{itemize}

\smallsection{Next steps}
What do researchers need to explore, in order to clarify the
similarities/differences among mobile release engineering vs non-mobile
release engineering:

\begin{itemize}
\tightlist
\item
  In one dimension:

  \begin{itemize}
  \tightlist
  \item
    Package manager
  \item
    App store (mobile)
  \item
    App store (non mobile, e.g., Steam)
  \end{itemize}
\item
  In another dimension:

  \begin{itemize}
  \tightlist
  \item
    Stakeholders (users and devs)
  \item
    Pipeline (control gatekeeper, speed, frequency)
  \item
    Feedback (issue tracking)
  \item
    Economics (revenue/models, market share, exclusivity)
  \end{itemize}
\end{itemize}

\subsubsection{Group 3. What Makes Mobile Special for Releng?}
\smallsection{Participants/Authors}
Meiyappan Nagappan, Julian Harty, Maur\'{i}cio Aniche, Luca Pascarella, Weiyi Shang

\smallsection{Differences between App Store and Package
Manager}
We analyzed the main differences between those two providers. In case of
App store the focus is on releasing application for general users, while
the package manager focuses on releasing libraries for a limited
audience (e.g., CS users).

During our discussion we realized that app stores are conceptually an
extension of package managers. The main features provided together with
the app stores are:

\begin{itemize}
\tightlist
\item
  More direct interaction with the users
\item
  There are several ranking criteria for users (e.g., downloads, scores)
\item
  Mobile Apps are expected to provide revenues
\item
  Libraries packaged together with the mobile app cannot be downloaded
\item
  The approval process is more strict
\item
  Mobile apps require permission from the user
\item
  One point of sale
\item
  Large population and more diverse users
\end{itemize}

This motivates the need for addressing the following challenges in
release engineering for mobile apps.

\smallsection{Challenges for Mobile-app release
engineering}

\begin{itemize}
\tightlist
\item
  Design testing strategies for non-functional requirements. Especially
  in case of small companies, testing for accessibility, usability, and
  energy consumption might be challenging.
\item
  Setting-up the testing infrastructure is more difficult to the
  fragmentation
\item
  A/B Testing is more complex
\item
  To address complaints in the reviews there is the need for releasing
  fast. However, the approval process is quite slow and prevent rapid
  releases.
\item
  Because of the need for super-fast releases, typicall development
  activites as code review need to be improved.
\item
  Traceability bugs \textless{}-\textgreater{} bad reviews.
\end{itemize}

\subsubsection{Consensus}

\begin{itemize}
\tightlist
    \item Advantages and disadvantages of a slow/fast release process
    \item Do not know what devs are doing
    \begin{itemize}
    \tightlist
        \item OSS/companies we talk to may not be representative
    \end{itemize}
    \item Test infrastructure
    \begin{itemize}
    \tightlist
        \item Fast lane $\rightarrow$ pipelines for app store
    \end{itemize}
    \item What is the effect of gate keeper?
    \begin{itemize}
    \tightlist
        \item Adv: gets usage data / easy movement of money
        \item Disadvantage: mercy of the gate keeper
    \end{itemize}
\end{itemize}

\subsection{Session 3: The State of the Practice}

\subsubsection{Continuous Experimentation for Mobile}


\smallsection{Participants/Authors}
Mei, Carmine, Sebastian, Shane, Tom, Raula, Keheliya, Afnan

\textbf{Continuous Experimentation}

\begin{itemize}
\item`whether you built the right thing (validation)`
\item *Testing:* whether you built the thing right (verification)
\end{itemize}

\textbf{Processes}
\begin{itemize}
\item A/B Testing
\item Canary Releases
\item Dark launches
\end{itemize}

\textbf{Mechanisms}
\begin{itemize}
\item Blue/Green
\item Feature toggles
\item Alpha users
\end{itemize}

\textbf{State-of-the-art}
\begin{itemize}
\item Third-party libraries
\item App store (i.e., gatekeepers) support (i.e., location based releases)
\end{itemize}

\textbf{Challenges}
\begin{enumerate}
\item How to do all these mechanisms in Mobile App?
\item Because there is no native support (by gatekeeper), can third-parties or gatekeeper be trusted, and what information needs?
\item Can micro-transactions (by gatekeeper) be used for continuous experimentation?
\item Are the processes mapping reviews+metrics to the deployed version?
\item How to architect for allowing dynamic views?
\item How to effectively use feature toggles with gatekeepers?
\item What are the challenges to collecting telemetry?
\item Continuous experimentation for the ~~losers~~masses?
\item How to do rollbacks with gatekeeper control?
\end{enumerate}

\textbf{Other challenges}
\begin{itemize}
\item Deploying hot fixes
\end{itemize}

\subsubsection{Quality Assurance in the Realm of Release Engineering of Mobile Apps}

\textbf{How does industry define QA?}

\begin{itemize}
\item Check: basically static analysis
\item Testing: more like exercising the app

\item Artifacts that need to be tested:
	\begin{itemize}
	\item Software code
	\item Resource files
	\item Unit tests
	\item Test in platform (by means of an emulator)
	\item Many devices
	\end{itemize}
\end{itemize}

\textbf{What are quality attributes for mobile?}

\begin{itemize}
\item Number of bugs, as it was always done?
\item Can we use \emph{user rating} or \emph{satisfaction} as quality attribute?
	\begin{itemize}
	\item Is that what we should care more about?
	\item Why similar apps have higher ratings than others?
	\item What do an app need to be successful? What does it mean 'to be successful'?
	\end{itemize}
\item Performance
\item Energy consumption
\end{itemize}

\textbf{Overall QA pipeline}

\begin{enumerate}
\item Static analysis
\item Prediction
\item Code reviews
\item User reviews
\item Automated testing
\item (to be continued)
\end{enumerate}

\textbf{Static analysis for mobile apps}

\begin{itemize}
\item Can we use existing tools, like Findbugs for mobile? Yes.
\item Specific for Android: *AndroBugs*. Vulnerability scanner, linters.
\item Still not enough, lots of false positives. Use of basic/simple rules.
	\begin{itemize}
	\item This is a general problem in static analysis (not only for mobile)
	\item Android's linter has accessibility checks.
	\end{itemize}
\item Do we have a tool that statically checks for energy consumption, accessibility, or other non-functional requirements?
\item Challenges: obfuscation, multi-language development (e.g., in Android, codebases have Java and C) <-- maybe a challenge for black box testing?
\end{itemize}

\textbf{Open questions and challenges}

\begin{itemize}
\item What are the characteristics of non-functional bugs?
	\begin{itemize}
	\item We need patterns/anti-patterns for detecting energy consumption, accessibility, usability, etc.
	\item See paper by Luis Cruz et al. (EMSE special issue on mobile apps)
	\end{itemize}
\item How well can we develop static analysis to detect non-functional problems?
\item How can we design static analysis for resources?
	\begin{itemize}
	\item See paper by Suelen Goularte et al. (EMSE special issue on mobile apps)
	\end{itemize}
\item Can we design tools/techniques that are platform-agnostic, or do they need to be platform-specific?
\end{itemize}


\textbf{Prediction}

\begin{itemize}
\item Do we need defect prediction techniques? They are small files, so maybe you need less prioritization.
	\begin{itemize}
	\item Maybe we need more defect localization, rather than prediction.
	\end{itemize}
\end{itemize}

\textbf{Code reviews}

\begin{itemize}
\item We can't see differences when it comes to code review in traditional apps.
\item RQ: What is different when reviewing mobile apps? Do they actually do it?
	\begin{itemize}
	\item Maybe related to the non-functional differences we discuss before.
	\end{itemize}
\item In a lot of small apps, there's just one developer. Is there a need for code review?
	\begin{itemize}
	\item Self-review might still be useful. Ex: if you program in VS Code, and review it on Gerrit, the difference in the IDE might help you in seeing things you don't.
	\end{itemize}
\end{itemize}

\textbf{Leveraging user reviews}

\begin{itemize}
\item How to extract bug reports from user reviews?
	\begin{itemize}
	\item Lots of papers on this topic already
	\end{itemize}
\item User reviews are noisy
	\begin{itemize}
	\item We need NLP approaches to remove noise
	\end{itemize}
\end{itemize}

\textbf{Testing}

\begin{itemize}
\item Challenges:
	\begin{itemize}
	\item Fragmentation
	\item (Environment) dependencies
	\item Interactions with other devices
	\end{itemize}

\item Tools exist for UI testing, such as Facebook's one
	\begin{itemize}
	\item Monkey testing is still the best (by Alex Orso)
	\end{itemize}

\item From a pragmatic point of view:
	\begin{itemize}
	\item Apps are made of a lot of UI code, which is hard to unit test.
	\item When to do unit testing or when to do system tests for Android?
	\item Mocking is a way, but in a UI-heavy application, do we want to mock the UI?
	\item RQ: What kind of tests do developers really \emph{do} in practice? What kind of tests they really \emph{need} in practice?
	\end{itemize}
	
\item Can we do (automatic) crash reproduction? 
	\begin{itemize}
	\item Julian: We would be ok if the reproduced crash is not an end-to-end. If it's a "unit test" (one, two, three classes), that'd be already useful for the developer to start working on the issue. 
	\end{itemize}
\end{itemize}

\textbf{Big questions}

\begin{itemize}
\item Should we help the 'big app developers' or the 'small app developer' (who are a huge part of the market)?
\item Can we learn the best QA practices from the data/developers we have available?
\item Given limited resources, what type of tests should we do?
\end{itemize}

\subsection{Session 4 and 5: Seeding Collaborative Projects}

Plenty of productive discussions were had during the first three sessions. In the last two sessions, we chose to focus on three topics that will seed future collaborative projects.

\subsubsection{Continuous Experimentation with Mobile App Restrictions}

\smallsection{Motivation}
To mitigate the risk of delivering buggy features to users, developers adopt incremental releasing strategies known as Continuous Experimentation (CE). CE also helps developers to determine if the new features are successful in serving the business goals, refining them if needed. Due to the app-store-centered delivery of mobile applications, CE in the mobile domain presents different challenges and opportunities than other SE disciplines. However, the extent to which CE is adopted in the mobile app development context is not well-understood.

\vspace{2mm}
\smallsection{Research Questions}

\begin{enumerate}[\bfseries RQ1]
	\item What are the common CE strategies practiced by mobile app developers?
	\item How are current tools supporting the practice of CE in mobile?
	\item What are the limitations and challenges in the state-of-the-art CE strategies and tools?
\end{enumerate}

\smallsection{Initial Approach}
We plan to conduct a mixed method study to understand the state of practice in Continuous Experimentation among mobile developers. First, a developer survey will provide insights on the level of maturity of CE practices, the limitations that developers face and the tooling that developers are using to solve the problems in CE. Based on the results from this qualitative study, we will mine the historical data in the source code repositories of mobile apps to quantify and characterize CE tool usage in practice.

\subsubsection{The Ubiquity of the App Store Paradigm (Beyond Mobile Apps)}

\smallsection{Motivation}
It quickly became clear during the meeting that the App Store paradigm has farther reaching implications than were anticipated prior to the meeting.
App stores have permeated a large number of sectors of the software market.
Users now expect to interact with app stores as the standard way to acquire software.
However, the implications of variations of app stores are not well understood.

\vspace{2mm}
\smallsection{Research Questions}

\begin{enumerate}[\bfseries RQ1]
	\item What are the emergent patterns in app stores?
	\item What are the implications of different app store patterns on stakeholders (e.g., developers, release engineers)?
\end{enumerate}

\smallsection{Initial Approach}
We plan to apply qualitative research methods to systematically curate a catalog of patterns of app stores and their perceived implications for stakeholders. This catalog will be a useful resource for more than just addressing our research questions and will be made openly available to the research community.

\subsubsection{Centralized Logging for Mobile Apps}

\smallsection{Motivation}
Logging is one of the main resources of information when software is running by the end users. Similarly, mobile app developers would also benefit from the logging information that provides to understand the quality of their apps, and the behaviour of the software in end users' mobile devices. However, the traditional logging mechanism makes it difficult for developers to collect the information by logs. Recent years, the advances of centralized logging infrastructure, like firebase, has made tremendous support on the applicably of having logging information available for the developers. Yet, the practices and the challenges on the use of these centralized logging infrastructure is not well-understood. 

\vspace{2mm}
\smallsection{Research Questions}

\begin{enumerate}[\bfseries RQ1]
\item To what extent does centralized logging infrastructure adopted by mobile app developers?
\item What are the information collected by centralized logging infrastructure?
\item What are the limitations and challenges in the state-of-the-art usage of centralized logging on mobile apps?
\end{enumerate}

\smallsection{Approach}
We have started to conduct an empirical study on the use of centralized logging infrastructure on open-source projects. In particular, we focus on the use of firebase centralized logging infrastructure and search the entire github for projects that leverage firebase. As a preliminary results, we have found over 100 open source projects that leverage the benefit of centralized logging infrastructure with firebase. By studying the usage of firebase, we find that developers often do not directly use firebase's API to log information. Instead, a customized logging utility class is often developers to support the ease of log for such centralized logging infrastructure. For next steps, we plan to study the rationale of developers adopting centralized logging infrastructure during their development history and making automated tool supports for developers based on our findings.

\clearpage

\section{List of Attendees}

\subsection{Co-organizers}

\begin{itemize}
\item Shane McIntosh (McGill University, Canada)
\item Yasutaka Kamei (Kyushu University, Japan)
\item Meiyappan Nagappan (University of Waterloo, Canada)
\end{itemize}

\subsection{Meeting Participants}

\begin{itemize}
\item Afnan Al-Subaihin (King Saud University, Saudi Arabia)
\item Maur\'icio Aniche (Delft University of Technology, Netherlands)
\item Daniel Dominguez (IMDEA Software Institute, Spain)
\item Keheliya Gallaba (McGill University, Canada)
\item Cuiyun Gao (Nanyang Technology University, Singapore)
\item Daniel M. German (University of Victoria, Canada)
\item Mike Godfrey (University of Waterloo, Canada)
\item Julian Harty (Commercetest Limited/Open University, UK)
\item Toshiki Hirao (Nara Institute of Science and Technology, Japan)
\item Masanari Kondo (Kyoto Institute of Technology, Japan)
\item Raula Gaikovina Kula (NAIST, Japan)
\item Li Li (Monash University, Australia)
\item Fabio Palomba (University of Zurich, Switzerland)
\item Luca Pascarella (Delft University of Technology, Netherlands)
\item Sebastian Proksch (University of Zurich, Switzerland)
\item Weiyi Shang (Concordia University, Canada)
\item Chakkrit (Kla) Tantithamthavorn (Monash University, Australia)
\item Patanamon (Pick) Thongtanunam (University of Melbourne, Australia)
\item Carmine Vassallo (University of Zurich, Switzerland)
\item Lili Wei (The Hong Kong University of Science and Technology, China)
\item Thomas Zimmermann (Microsoft Research, USA)
\end{itemize}

\end{document}
